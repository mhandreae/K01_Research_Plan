\documentclass[]{article}
\usepackage{lmodern}
\usepackage{amssymb,amsmath}
\usepackage{ifxetex,ifluatex}
\usepackage{fixltx2e} % provides \textsubscript
\ifnum 0\ifxetex 1\fi\ifluatex 1\fi=0 % if pdftex
  \usepackage[T1]{fontenc}
  \usepackage[utf8]{inputenc}
\else % if luatex or xelatex
  \ifxetex
    \usepackage{mathspec}
    \usepackage{xltxtra,xunicode}
  \else
    \usepackage{fontspec}
  \fi
  \defaultfontfeatures{Mapping=tex-text,Scale=MatchLowercase}
  \newcommand{\euro}{€}
\fi
% use upquote if available, for straight quotes in verbatim environments
\IfFileExists{upquote.sty}{\usepackage{upquote}}{}
% use microtype if available
\IfFileExists{microtype.sty}{%
\usepackage{microtype}
\UseMicrotypeSet[protrusion]{basicmath} % disable protrusion for tt fonts
}{}
\usepackage[margin=1in]{geometry}
\ifxetex
  \usepackage[setpagesize=false, % page size defined by xetex
              unicode=false, % unicode breaks when used with xetex
              xetex]{hyperref}
\else
  \usepackage[unicode=true]{hyperref}
\fi
\hypersetup{breaklinks=true,
            bookmarks=true,
            pdfauthor={},
            pdftitle={},
            colorlinks=true,
            citecolor=blue,
            urlcolor=blue,
            linkcolor=magenta,
            pdfborder={0 0 0}}
\urlstyle{same}  % don't use monospace font for urls
\setlength{\parindent}{0pt}
\setlength{\parskip}{6pt plus 2pt minus 1pt}
\setlength{\emergencystretch}{3em}  % prevent overfull lines
\setcounter{secnumdepth}{0}

%%% Use protect on footnotes to avoid problems with footnotes in titles
\let\rmarkdownfootnote\footnote%
\def\footnote{\protect\rmarkdownfootnote}

%%% Change title format to be more compact
\usepackage{titling}
\setlength{\droptitle}{-2em}
  \title{}
  \pretitle{\vspace{\droptitle}}
  \posttitle{}
  \author{}
  \preauthor{}\postauthor{}
  \date{}
  \predate{}\postdate{}




\begin{document}

\maketitle


\section{Research Plan}\label{research-plan}

\subsection{Significance}\label{significance}

\subsubsection{Clincial Impact}\label{clincial-impact}

\subparagraph{Acute respiratory failure is a significant burden of
disease.}\label{acute-respiratory-failure-is-a-significant-burden-of-disease.}

Many hospitalized patients develop acute respiratory failure (SHINYstan
Team, 2015), which is worrisome.

\subparagraph{more respiratory failure is a significant burden of
disease.}\label{more-respiratory-failure-is-a-significant-burden-of-disease.}

Indeed ny hospitalized patients develop acute respiratory failure
(SHINYstan Team, 2015), which is worrisome.

\subsubsection{Bayesian imputation}\label{bayesian-imputation}

\paragraph{Heterogeneous provider compliance and missing clinical data
may limit
implementation}\label{heterogeneous-provider-compliance-and-missing-clinical-data-may-limit-implementation}

of the prediction algorithm, the therapeutic interventions and the trial
itself. Variables with strong predicitive power in our model may not be
recorded in all patients or may be missing for the time window needed
for prediction. To improve prediction for cases with incomplete data we
can impute the missing data. Likelihood-based mixed effects models for
incomplete data give valid estimates when the data are ignorably
missing; that is, the parameters for the missing data process are
distinct from those of the main model for the outcome, and the data are
missing at random (MAR). This unlikely in our setting of clinical data
in the EMR, because physicians will request test based on the patients
comorbidities and current clinical conditions. Data will not be missing
at random, but incompleteness will be associated with predictors and
outcomes.

\subparagraph{Incomplete data can be imputed from auxilary
data,}\label{incomplete-data-can-be-imputed-from-auxilary-data}

that is additional information available in the form of an auxiliary
variable known to be correlated with the missing outcome of interest.
For example, arterial blood gas oxygen saturation may be used to impute
peripheral pulse oxymetry or oxygen therapy, if the latter are
unavailable for the prediction time window, and vice versa.

first (Wang \& Hall, 2010). second(Hall, Lipton, Katz, \& Wang, 2014)

Joint modeling methods can help address bias caused by informative
missing data in the estimation of the effect of{]}

Hall, C. B., Lipton, R. B., Katz, M. J., \& Wang, C. (2014). Correcting
bias caused by missing data in the estimate of the effect of
apolipoprotein epsilon 4 on cognitive decline. \emph{J Int Neuropsychol
Soc}, 1--6.
doi:\href{http://dx.doi.org/10.1017/S1355617714000952}{10.1017/S1355617714000952}

SHINYstan Team. (2015). SHINYstan: R package for interactive exploration
of markov chain monte carlo output, version 0.1. Retrieved from
\url{https://github.com/jgabry/SHINYstan}

Wang, C., \& Hall, C. B. (2010). Correction of bias from non-random
missing longitudinal data using auxiliary information. \emph{Stat Med},
\emph{29}(6), 671--679.
doi:\href{http://dx.doi.org/10.1002/sim.3821}{10.1002/sim.3821}

\end{document}
