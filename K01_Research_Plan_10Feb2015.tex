\documentclass[]{article}
\usepackage{lmodern}
\usepackage{amssymb,amsmath}
\usepackage{ifxetex,ifluatex}
\usepackage{fixltx2e} % provides \textsubscript
\ifnum 0\ifxetex 1\fi\ifluatex 1\fi=0 % if pdftex
  \usepackage[T1]{fontenc}
  \usepackage[utf8]{inputenc}
\else % if luatex or xelatex
  \ifxetex
    \usepackage{mathspec}
    \usepackage{xltxtra,xunicode}
  \else
    \usepackage{fontspec}
  \fi
  \defaultfontfeatures{Mapping=tex-text,Scale=MatchLowercase}
  \newcommand{\euro}{€}
\fi
% use upquote if available, for straight quotes in verbatim environments
\IfFileExists{upquote.sty}{\usepackage{upquote}}{}
% use microtype if available
\IfFileExists{microtype.sty}{%
\usepackage{microtype}
\UseMicrotypeSet[protrusion]{basicmath} % disable protrusion for tt fonts
}{}
\usepackage[margin=1in]{geometry}
\ifxetex
  \usepackage[setpagesize=false, % page size defined by xetex
              unicode=false, % unicode breaks when used with xetex
              xetex]{hyperref}
\else
  \usepackage[unicode=true]{hyperref}
\fi
\hypersetup{breaklinks=true,
            bookmarks=true,
            pdfauthor={},
            pdftitle={},
            colorlinks=true,
            citecolor=blue,
            urlcolor=blue,
            linkcolor=magenta,
            pdfborder={0 0 0}}
\urlstyle{same}  % don't use monospace font for urls
\setlength{\parindent}{0pt}
\setlength{\parskip}{6pt plus 2pt minus 1pt}
\setlength{\emergencystretch}{3em}  % prevent overfull lines
\setcounter{secnumdepth}{0}

%%% Use protect on footnotes to avoid problems with footnotes in titles
\let\rmarkdownfootnote\footnote%
\def\footnote{\protect\rmarkdownfootnote}

%%% Change title format to be more compact
\usepackage{titling}
\setlength{\droptitle}{-2em}
  \title{}
  \pretitle{\vspace{\droptitle}}
  \posttitle{}
  \author{}
  \preauthor{}\postauthor{}
  \date{}
  \predate{}\postdate{}

\setlength{\parskip}{0pt}
\setlength{\parsep}{0pt}


\begin{document}

\maketitle


\section{Research Plan}\label{research-plan}

\subsection{Significance}\label{significance}

\subsubsection{Clincial Impact}\label{clincial-impact}

\subparagraph{Acute respiratory failure is a significant burden of
disease.}\label{acute-respiratory-failure-is-a-significant-burden-of-disease.}

Many hospitalized patients develop acute respiratory failure (SHINYstan
Team, 2015), which is worrisome.

\subsection{Hierarchical modeling exploits the rich heterogeneity of
electronic medical
records}\label{hierarchical-modeling-exploits-the-rich-heterogeneity-of-electronic-medical-records}

\subparagraph{Electronic medical records are an eminent example of Big
Data.}\label{electronic-medical-records-are-an-eminent-example-of-big-data.}

EMR have more useful data than can be analyzed more useful data than can
be analyzed in a scientifically meaningful way by existing statistical
inference tools. This limiting the scientific hypotheses and clinical
inferences, that can be explored and evaluated. Large electronic medical
data sets are not just bigger in that there are more instances of the
same thing, (this would make data analysis only easier). Rather, there
is more breadth to the data: more subgroups, locations, or time
granularity than is currently being modeled, more partial and noisy
measurements that cannot easily be incorporated into standard models,
more information on the population units being measured, and more
fine-grained information on the predictions desired. EMR are the prime
example of richly structured and correlated web of data.

\subparagraph{Big data like electronic medical records are nested
hierarchically.}\label{big-data-like-electronic-medical-records-are-nested-hierarchically.}

Clinical observations are nested within patients, e.g.~repeated glucose
measurements will be similar in the same patients. Patients seen by the
same provider will have similar outcomes predicted by provider behavior
and qualities. Providers are integrated in institutions. Institutions
are nested geographically in counties and regions. Healthcare
environments predict patient and provider behavior and outcomes.
Patients seen by the same team, treated in the same setting will have
similar propensity to respond to interventions. Big data requires more
than just fitting well-known models at larger scales; it requires richer
models to exploit fine-grained multilevel structures and to map to
predictive questions of interest.

\subparagraph{Bayesian hierarchical modeling of complex Big Data is
transformative.}\label{bayesian-hierarchical-modeling-of-complex-big-data-is-transformative.}

With its flexibility and robustness Bayesian models may predict better
in large data sets with spatial and temporal organization, than
classical models (Gelman, 2009). Consider our multilevel electronic
medical records dataset consisting of repeated visits by patients with
different ages and medical conditions in different services integrated
in different hospitals in different states with different medical plans.
Fitting the predictive regression model, we would want the regression
coefficients to vary by group (by service, by medical unit, by
hospital), to realistically model the complex correlations seen in
actual clinical practice: The number of parameters to estimate grows
very quickly and so do the potential interactions. Reciprocally, even
with very large datasets, the sample size in each subgroup will shrink
rapidly; estimates using least squares or maximum likelihood will become
noisy and thus often become essentially useless. Regardless, we will
want to estimate various hyperparameters and hyper-hyperparameters, to
represent how lower level parameters vary across different groupings
(Bafumi \& Gelman, 2007).

\subparagraph{``Partial pooling'' outperforms the no-pooling and
complete-pooling
alternatives.}\label{partial-pooling-outperforms-the-no-pooling-and-complete-pooling-alternatives.}

Hierarchical modeling is more efficient, as can be shown mathematically
or via cross-validation (Gelman, Carlin, Stern, \& Rubin, 2014). ``No
pooling''" is one approach to estimate the model for each group
separately. Addressing and exploring the complexity and granularity, the
richness of the data may lead to far too many subclassifications, thus
too small samples in any given subgroup for useful inferences.
``Complete pooling'' or structural modeling is another approach, but the
implied hard constraints on the coefficients in different groups may
lead to bias, in particular for groups with sparse data; we loose
information, because we cannot learn from groups where we have more
data. In hierarchical modeling, the estimate of each individual
parameter is simultaneously informed by data from all the other units;
this is what makes ``partial pooling'' or hierarchical modeling
especially effective (Gelman, 2006).

\paragraph{Heterogeneous and incomplete clinical data may limit
prediction and
implementation.}\label{heterogeneous-and-incomplete-clinical-data-may-limit-prediction-and-implementation.}

Variables with strong predicitive power in our model may not be recorded
in all patients or may be missing for the time window needed for
prediction, limiting development of the prediction algorithm,
implementation of the therapeutic interventions and the trial itself. To
improve prediction for cases with incomplete data, we can impute the
missing data. Informative loss by incomplete data may bias risk
prediction or may hamper the implementation of the prediction algorithm.
Likelihood-based mixed effects models for incomplete data give valid
estimates if and only if the data are ignorably missing; that is, the
parameters for the missing data process are distinct from those of the
main model for the outcome, and the data are missing at random (MAR)
(Rubin, 1976). However, this is an unreasonable assumption for our
electronic medical records, for example because physicians will request
test based on the patients comorbidities and current clinical
conditions. Data will not be missing at random, instead incomplete data
will be associated with predictors and outcomes.

\subparagraph{Developing new Bayesian methods for imputation of
incomplete data from auxilliary
data.}\label{developing-new-bayesian-methods-for-imputation-of-incomplete-data-from-auxilliary-data.}

Auxilliary date are additional information available in the form of
variables known to be correlated with the missing data of interest. For
example, arterial blood gas oxygen saturation may be used to impute
peripheral pulse oxymetry or oxygen therapy, if the latter are
unavailable for the prediction time window, and vice versa. This
appraoch avoids the perils associated with missing at random (MAR)
assumptions, when fitting a non-ignorable missingness model (Wang \&
Hall, 2010). Adding auxiliary variables not included in the main model
for multiple imputation, in other words using additional information
that is correlated with the missing outcome is an emerging approach to
help correct bias (Collins, Schafer, \& Kam, 2001; Meng, 1994; Rubin,
1996), often relying on Bayesian methods for the multiple imputations
approach (Daniels \& Hogan, 2008; J. L. Schafer, 1997); joint modeling
and multiple imputations could both be used also to impute incomplete
medical records (Fitzmaurice, Davidian, Verbeke, \& Molenberghs, 2008).
The use of auxiliary data to impute incomplete patient records will
improve the prediction model and facilitate smoother implementation of
the algorithm into the clinical trial (Hall, Lipton, Katz, \& Wang,
2014). Moreover, auxilliary data imputation for incomplete electronic
medical records is underdeveloped; methodologically, their development
is an innovative hallmark of this proposal.

\subparagraph{Integration of seasonal and secular effects, provider
compliance and institutional
learning}\label{integration-of-seasonal-and-secular-effects-provider-compliance-and-institutional-learning}

Non-compliance is a major obstacle to the effective delivery of health
care and improved outcomes (Duncan, McIntosh, Stayton, \& Hall, 2006).
The personalized intervention triggered by our EMR-prediction algorithm
will only prevent respiratory failure if our physicians and nurses
implement them. Improving compliance of healthcare providers with
evidence based interventions continues to be a challenge and is
under-researched (Davis, Thomson, Oxman, \& Haynes, 1995). Institutional
behavior changes in response to trials and quality improvments
interventions; patient populations can change over time. Respiratory
patients are plagued by seasonal deteriorations, which could lead to
bias in our model. These seasonal and secular effects will alter the
predictors of risk in our model. We will therefore include seasonal
effects and continuously update our model with new patient data during
the implementation of our trial to account for said changes in the risk
profile. The integration of provider compliance, secular and seasonal
effects in a EMR triggered pragamatic trial with one coherent model is
novel.

\section{Impact}\label{impact}

\subparagraph{Focus on prevention of critical adverse outcomes in
hospitalized
patients.}\label{focus-on-prevention-of-critical-adverse-outcomes-in-hospitalized-patients.}

Changes in reimbursement give providers a stake in patient outcomes and
led to a keen interest in the prediction and prevention of adverse event
in hospitalized patients. It makes sense to focus early intervention on
patients at high risk for poor outcomes. This project advances
hierarchical Bayesian models to implement this paradigm shift in very
large electronic medical records, triggering personalised interventions
that drive outcome improvements.

\subparagraph{Further develop Bayesian methods to impute incomplete
electronic medical
records}\label{further-develop-bayesian-methods-to-impute-incomplete-electronic-medical-records}

Incomplete patient data, typical for actual clincal records can hinder
the development and execution of our prediction algorithms. We will
utilize additional auxilliary data to impute incomplete missing data to
overcome this limitation. Beyond improving prediction and patient
outcomes in our clinical trial, the Bayesian methods we propose to
develop can be employed in other settings to impute incomplete
electronic medical records.

\subparagraph{Integration of critical care with computational
statistics}\label{integration-of-critical-care-with-computational-statistics}

To often advances in statistical modeling and clinical science are
worlds apart. We address a critical problem of scalability in Big Data
inference, but are equally motivated by our practical use case,
improving our pragmatic clinical trial. The strength of our proposal,
however, is the integration of disparate disciplines, critical care and
computational statistics.

\subsection{Summary of the impact}\label{summary-of-the-impact}

We tackle a serious health care challenge by integrating advanced
hierarchical modeling into a pragmatic clinical trial. Beyond improving
morbidity and mortality from respiratory disease in hospitalized
patients through improved prediction and prevention, we will develop new
methods to impute incomplete electronic medical records from auxillary
data and scale Bayesian hierachical models to use in large EMR data. Our
proposal is unique and novel in its integration of cutting edge methods
from clinical, statistical and computer science to fully realize the
promise of Big Data in critical care.

\section*{References}\label{references}
\addcontentsline{toc}{section}{References}

Bafumi, J., \& Gelman, A. (2007). \emph{Fitting multilevel models when
predictors and group effects correlate} (SSRN Scholarly Paper No. ID
1010095). Rochester, NY: Social Science Research Network. Retrieved from
\url{http://papers.ssrn.com/abstract=1010095}

Collins, L. M., Schafer, J. L., \& Kam, C. M. (2001). A comparison of
inclusive and restrictive strategies in modern missing data procedures.
\emph{Psychol Methods}, \emph{6}(4), 330--351.

Daniels, M. J., \& Hogan, J. W. (2008). \emph{Missing data in
longitudinal studies: Strategies for bayesian modeling and sensitivity
analysis} (pp. --). CRC Press.

Davis, D. A., Thomson, M. A., Oxman, A. D., \& Haynes, R. B. (1995).
Changing physician performance. a systematic review of the effect of
continuing medical education strategies. \emph{JAMA}, \emph{274}(9),
700--705.

Duncan, M. M., McIntosh, P. A., Stayton, C. D., \& Hall, C. B. (2006).
Individualized performance feedback to increase prenatal domestic
violence screening. \emph{Matern Child Health J}, \emph{10}(5),
443--449.
doi:\href{http://dx.doi.org/10.1007/s10995-006-0076-0}{10.1007/s10995-006-0076-0}

Fitzmaurice, G., Davidian, M., Verbeke, G., \& Molenberghs, G. (2008).
\emph{Longitudinal data analysis}. CRC Press.

Gelman, A. (2006). Multilevel (hierarchical) modeling: What it can and
cannot do. \emph{Technometrics}, \emph{48}(3), 432--435.
doi:\href{http://dx.doi.org/10.1198/004017005000000661}{10.1198/004017005000000661}

Gelman, A. (2009). \emph{Red state, blue state, rich state, poor state:
why americans vote the way they do}. Princeton University Press.

Gelman, A., Carlin, J. B., Stern, H. S., \& Rubin, D. B. (2014).
\emph{Bayesian data analysis} (Vol. 2). Taylor \& Francis.

Hall, C. B., Lipton, R. B., Katz, M. J., \& Wang, C. (2014). Correcting
bias caused by missing data in the estimate of the effect of
apolipoprotein epsilon 4 on cognitive decline. \emph{J Int Neuropsychol
Soc}, 1--6.
doi:\href{http://dx.doi.org/10.1017/S1355617714000952}{10.1017/S1355617714000952}

Meng, X.-L. (1994). Multiple-imputation inferences with uncongenial
sources of input. \emph{Statist. Sci.}, \emph{9}(4), 538--558.
doi:\href{http://dx.doi.org/10.1214/ss/1177010269}{10.1214/ss/1177010269}

Rubin, D. B. (1976). Inference and missing data. \emph{Biometrika},
\emph{63}(3), 581--592.

Rubin, D. B. (1996). Multiple imputation after 18+ years. \emph{Journal
of the American Statistical Association}, \emph{91}(434), 473--489.

Schafer, J. L. (1997). \emph{Analysis of incomplete multivariate data}.
CRC press.

SHINYstan Team. (2015). SHINYstan: R package for interactive exploration
of markov chain monte carlo output, version 0.1. Retrieved from
\url{https://github.com/jgabry/SHINYstan}

Wang, C., \& Hall, C. B. (2010). Correction of bias from non-random
missing longitudinal data using auxiliary information. \emph{Stat Med},
\emph{29}(6), 671--679.
doi:\href{http://dx.doi.org/10.1002/sim.3821}{10.1002/sim.3821}

\end{document}
